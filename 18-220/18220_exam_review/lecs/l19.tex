\subsection*{Lecture 19}
Operating circuits at AC around a DC operating point, small signal analysis.

\begin{enumerate}
  %% Problem defines
  \def\diodeCurrent{\SI{30}{\milli\ampere}}
  \def\opPtA{\SI{5}{\volt}}
  \def\opPtB{\SI{12}{\volt}}
  \item Given the following circuit,
  \begin{figure}[H]
    \centering
    \def\basex{0}
    \def\basey{0}
    \def\stretch{1.5}
    \def\width{2}
    \begin{circuitikz}
      \draw 
        (0,\basey) node[sground] {} to[sV, l=$\vin$]
        (0,\basey+\stretch) to[V, l=$\VDC$, invert]
        (0,\basey+2*\stretch) to[short]
        (\basex+\width,\basey+2*\stretch) to[R=$R$]
        (\basex+\width,\basey+\stretch) to[D*, l=$D$]
        (\basex+\width,\basey) node[sground] {}
      ;
    \end{circuitikz}
    \label{l19:diodeLinear}
  \end{figure}
  Assume the diode can be modified with the standard diode equation,
  with $\VT = \SI{0.026}{\volt}, \IS=\SI{1e-12}{\ampere}$
  \begin{enumerate}
    \item If $\VDC = \opPtA$, find the required $R$ to achieve a bias current of
    \diodeCurrent through the diode.
    \item At this DC operating point, what is the change in current with respect to the
    change in voltage?
    \item What is the change in current with respect to the change in the \textit{operating voltage point}?
    \item If we want to increase $\VDC = \opPtB$ and still maintain \diodeCurrent
    through the diode, what resistor can we add in parallel to $R$ to get our desired operating point?
    \item We observed two operating points in this problem, $\VDC = \opPtA, \opPtB$.
    Which operating point gives a bigger current swing? 
  \end{enumerate}
\end{enumerate}